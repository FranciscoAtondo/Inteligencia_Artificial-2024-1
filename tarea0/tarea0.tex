%% -*- mode: latex; coding: utf-8; -*-
%% Comando para compilar: lualatex tarea0.tex

\documentclass[11pt,letterpaper]{article}


\usepackage{polyglossia}
\usepackage{fancyhdr}
\usepackage[margin=1in]{geometry}
\usepackage{algpseudocode}
\usepackage{amsthm}
\usepackage{framed}
\usepackage{hyperref}
\usepackage{tikz}
\usepackage[margin=1in]{geometry}
\usepackage{mathtools,amsthm}
\usepackage{enumitem,amssymb}
\usepackage{titling}
\usepackage[default]{fontsetup}


\setdefaultlanguage{spanish}
\setlength{\headheight}{13.6pt}
\setlength{\droptitle}{-11.5ex}
\definecolor{shadecolor}{gray}{0.925}
\newenvironment{solution}{%
  \noindent\begin{shaded}
  \textbf{Solución:}\ }{
  \end{shaded}%
}
\chead{}
\rhead{\today}
\lfoot{}
\cfoot{Inteligencia Artificial --- LCC 2024--I}
\rfoot{\thepage}
\renewcommand{\headrulewidth}{0.4pt}
\renewcommand{\footrulewidth}{0.4pt}
\pagestyle{fancy}
\setlength{\parindent}{0pt}


\newcommand{\bvec}[1]{\symbfit{#1}}
\newcommand{\norm}[1]{\left\lVert#1\right\rVert}


\title{%
  \bfseries
  Inteligencia artificial\\%
  Lic. en Ciencias de la Computación\\%
  Tarea 0
}
\date{}


\begin{document}

\maketitle

\vspace{-2.5cm}
\begin{center}
  \begin{tabular}{rl}
    Expediente: & 211214526 \\
    Nombre: & Francisco Javier Fernando Atondo Nubes \\
    Colaboradores:
  \end{tabular}
\end{center}

{\itshape Al entregar esta tarea, declaro que todas las respuestas son
  producto de mi propio trabajo y de las personas que colaboraron
  especificadas arriba.}


%%%%%%%%%%%%%%%%%%%%%%%%%%%%%%%%%
%% OPTIMIZACIÓN Y PROBABILIDAD %%
%%%%%%%%%%%%%%%%%%%%%%%%%%%%%%%%%

\section*{Optimización y probabilidad}

\begin{enumerate}
\item%
  Sean \(x_1,\dots,x_n\) números reales representando posiciones sobre
  una recta.  Sean \(w_1,\dots,w_n\) números reales positivos
  representando la \emph{importancia} de cada una de estas
  posiciones. Considera la función cuadrática,
  %% 
  \[ f(\theta) = \sum_{i=1}^n w_i\left(\theta-x_i\right)^2 \]
  %% 
  y que \(\theta\) es un escalar. ¿Qué valor de \(\theta\) minimiza
  \(f(\theta)\)? Muestra que el óptimo que encontraste es realmente un
  mínimo. ¿Qué cuestiones problemáticas pueden surgir si algunas de
  las \(w_i\) son negativas?
  \begin{solution}
    \[
    d_\theta f\left(\theta\right) = d_\theta ( \sum_{i=1}^n w_i (\theta-x_i)^2)\]
    \[ \longrightarrow \theta = \sum_{i=1}^n 2(\theta - x_i)\] 
    \[ \longrightarrow \theta = \sum_{i=1}^n 2\theta - \sum_{i=1}^n 2x_i\]
    
  \end{solution}
\item%
  Considera las siguientes funciones,
  %% 
  \[
  \begin{aligned}
    f(\bvec{x}) &= \min_{s\in[-1,1]}\sum_{i=1}^d sx_i \\
    g(\bvec{x}) &= \sum_{i=1}^d \min_{s_i\in[-1,1]} s_ix_i
  \end{aligned}
  \]
  donde \(\bvec{x} = (x_1,\dots,x_d)\in\mathbb{R}^d\) es un
  vector real y \([-1,1]\) el intervalo cerrado entre \(-1\) a
  \(1\). ¿Cuál de las siguientes desigualdades es cierta para toda
  \(\bvec{x}\)?  Demuéstralo.
  %% 
  \[
  \begin{aligned}
    f(\bvec{x}) &\leq g(\bvec{x}) \\
    f(\bvec{x}) &= g(\bvec{x}) \\
    f(\bvec{x}) &\geq g(\bvec{x})
  \end{aligned}
  \]
  %% 
  \begin{solution}
    %% Escribe tu solución aquí
  \end{solution}    
\item%
  Supongamos que lanzas repetidamente un dado justo de seis caras
  hasta que obtienes un resultado de \(1\) (y luego te detienes).
  Cada vez que lanzas un \(3\) ganas \(a\) puntos, y cada vez que
  lanzas un \(6\) pierdes \(b\) puntos.  No ganas ni pierdes puntos si
  lanzas un \(2\), \(4\) o \(5\).  ¿Cuál es la cantidad de puntos
  (como función de \(a\) y \(b\)) que esperamos tener cuando te
  detengas?
  %%
  \begin{solution}
    %% Escribe tu solución aquí
  \end{solution}
\item%
  Supongamos que la probabilidad de que una moneda caiga en águila es
  \(p\) (donde \(0 < p < 1\)), y que lanzas esta moneda cinco veces
  obteniendo \(\left(S, A, A, A, A\right)\).  Sabemos que la
  probabilidad de obtener esta secuencia es,
  %% 
  \[ L(p) = (1-p)pppp = p^4(1-p) \]
  %% 
  ¿Qué valor de \(p\) maximiza \(L(p)\)? Muestra que este valor de
  \(p\) maximiza \(L(p)\). ¿Cuál es una interpretación intuitiva de
  este valor de \(p\)?
  %% 
  \begin{solution}
    %% Escribe tu solución aquí
  \end{solution}
\item%
  Supongamos que \(A\) y \(B\) son dos eventos tales que \(P(A \mid B)
  = P(B \mid A)\).  También sabemos que \(P(A \cup B) = \frac{1}{3}\)
  y que \(P(A \cap B) > 0\).  Muestra que \(P(A) > \frac{1}{6}\).
  %% 
  \begin{solution}
    %% Escribe tu solución aquí
  \end{solution}
  
\item%
  Considera un vector columna \(\bvec{w}\in\mathbb{R}^d\) y vectores
  columna constantes \(\bvec{a}_i,\bvec{b}_j\in\mathbb{R}^d\),
  \(\lambda\in\mathbb{R}\) y un entero positivo \(n\).  Define la
  función con valor escalar,
  %% 
  \[ f(\bvec{w})=\left(\sum_{i=1}^n\sum_{j=1}^n\left(\bvec{a}_i^\top\bvec{w}-\bvec{b}_j^\top\bvec{w}\right)^2\right)+\frac{\lambda}{2}\norm{\bvec{w}}_2^2, \]
  %% 
  donde el vector es \(\bvec{w} = (w_1,\dots,w_d)^\top\) y
  \(\norm{\bvec{w}}_2 = \sqrt{\sum_{k=1}^d w_k^2} =
  \sqrt{\bvec{w}^\top\bvec{w}}\) es conocida como la norma \(L_2\).
  Calcula el gradiente \(\nabla f(\bvec{w})\).
  %% 
  \begin{solution}
    %% Escribe tu solución aquí
  \end{solution}
\end{enumerate}

\newpage
%%%%%%%%%%%%%%%%%%%%%%%%%%%%%%%
%% COMPLEJIDAD COMPUTACIONAL %%
%%%%%%%%%%%%%%%%%%%%%%%%%%%%%%%

\section*{Complejidad computacional}

\begin{enumerate}
\item%
  Supongamos que tienes una cuadrícula de puntos de \(n \times n\),
  donde nos gustaría colocar \(3\) rectángulos alineados a los ejes
  (los lados del rectángulo son paralelos a los ejes).  Cada esquina
  de cada rectángulo debe ser uno de los puntos en la cuadrícula, pero
  fuera de eso no hay restricciones sobre la ubicación o tamaño de los
  rectángulos.  Por ejemplo, es posible que las cuatro esquinas de un
  rectángulo estén en el mismo punto (resultando en un rectángulo de
  tamaño \(0\)), o que todos los \(3\) rectángulos estén encimados.
  ¿De cuántas maneras se pueden colocar los \(3\) rectángulos sobre la
  cuadrícula?  En general, solo nos importa la complejidad asintótica,
  entonces escribe tu respuesta de la forma \(O(n^c)\) o de la forma
  \(O(c^n)\) para algún entero \(c\).
  %% 
  \begin{solution}
  %%
     Como se trata de 3 rectangulos, y la posicion y tamaño de estas no importa, mientras que las esquinas de estos esten en la cuadricula, podemos usar dos esquinas opuestas de dos rectangulos diferentes para crear otro rectangulo, del cual seleccionamos otras 4 esquinas opuestas para crear otros dos rectangulos. Esto nos da un total de $n^2 * n^2 * n^2$, o $ n^6 $ opciones, el cual se debe dividir entre las formas de crear los rectangulos:
     \[ \frac{n^6}{3!} \]
     Lo cual se puede expresar Asintoticamente como: $ O(n^6) $
    %%\[ f(C) = \sum_{i=0}^n \]
  %%
  \end{solution}
\item%
  Supongamos que tienes una cuadrícula de puntos de \(n \times 2n\).
  Comenzamos en el punto de la esquina superior izquierda (el punto en
  la posición \((1,1)\)), y nos gustaría llegar al punto de la esquina
  inferior derecha (el punto en la posición \((n, 2n)\)) moviéndose
  exclusivamente o hacia abajo o hacia la derecha.  Supongamos que se
  nos provee una función \(c(i,j)\) que produce el costo asociado con
  la posición \((i,j)\), y supongamos que para cada posición toma
  tiempo constante calcular este costo.  El costo puede ser negativo.
  Define el costo de un camino como la suma de \(c(i,j)\) para todos
  los puntos \((i,j)\) sobre el camino, incluyendo ambos extremos.
  Presenta un algoritmo para calcular el costo del camino de costo
  mínimo desde \((1,1)\) hasta \((n,2n)\) de la manera más eficiente
  posible (con la complejidad en tiempo más pequeña).  ¿Cuál es el
  tiempo de ejecución?
  %% 
  \begin{solution}
    %% Escribe tu solución aquí
  \end{solution}
\end{enumerate}


%%%%%%%%%%%%%%%%%%%%%%%%%%%%
%% CONSIDERACIONES ÉTICAS %%
%%%%%%%%%%%%%%%%%%%%%%%%%%%%

\section*{Consideraciones éticas}

\begin{enumerate}
\item%
  Una empresa de inversión desarrolla un modelo simple de aprendizaje
  automático para predecir si es probable que un individuo incumpla
  con un préstamo a partir de una variedad de factores, incluida la
  ubicación, la edad, la puntuación crediticia y los registros
  públicos.  Después de examinar sus resultados, se encuentra que el
  modelo predice principalmente en función de la ubicación y que el
  modelo acepta principalmente préstamos de centros urbanos y niega
  préstamos a solicitantes rurales.  Además, al observar el género y
  el origen étnico de los solicitantes, se encuentra que el modelo
  tiene una tasa de falsos positivos significativamente mayor para los
  solicitantes negros y masculinos que para otros grupos.  En una
  predicción falsa positiva, un modelo clasifica erróneamente a
  alguien que no incumple como probable que incumpla.
  %% 
  \begin{solution}
    %% Escribe tu solución aquí
  \end{solution}
\item%
  La estilometría es una forma de predecir la autoría de un texto
  anónimo o impugnado, mediante el análisis de los patrones de
  escritura en el texto anónimo y otros textos escritos por los
  autores potenciales.  Recientemente, se han desarrollado algoritmos
  de aprendizaje automático de alta precisión para esta tarea.  Si
  bien estos modelos se utilizan normalmente para analizar documentos
  históricos y literatura, podrían usarse para desanonimizar una
  amplia gama de textos, incluido el código.
  %% 
  \begin{solution}
    %% Escribe tu solución aquí
  \end{solution}

\item%
  Un grupo de investigación analizó millones de rostros de
  celebridades de las imágenes de Google para desarrollar una
  tecnología de reconocimiento facial.  Las celebridades no dieron
  permiso para que sus imágenes se utilizaran en el conjunto de datos
  y muchas de las imágenes tienen derechos de autor.  Para fotografías
  con derechos de autor, el conjunto de datos proporciona enlaces URL
  a la imagen original junto con cuadros delimitadores para la cara.
  %% 
  \begin{solution}
    %% Escribe tu solución aquí
  \end{solution}
  
\item%
  Los investigadores han creado recientemente un modelo de aprendizaje
  automático que puede predecir especies de plantas automáticamente y
  directamente a partir de una sola fotografía.  El modelo fue
  entrenado usando fotografías cargadas en una aplicación por usuarios
  que dieron su consentimiento para usar sus fotografías con fines de
  investigación, y el modelo solo se usa dentro de la aplicación para
  ayudar a los usuarios a identificar plantas que podrían encontrar en
  la naturaleza.
  %% 
  \begin{solution}
    %% Escribe tu solución aquí
  \end{solution}
\end{enumerate}


%%%%%%%%%%%%%%%%%%
%% PROGRAMACIÓN %%
%%%%%%%%%%%%%%%%%%

\section*{Programación}

\begin{solution}
  Incorporada en \texttt{tarea0.py}.
\end{solution}

\end{document}
